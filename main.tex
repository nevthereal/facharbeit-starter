\documentclass[12pt, a4paper]{article}
\usepackage{graphicx}
\linespread{1.5}
\usepackage[ngerman]{babel}
\usepackage[colorlinks=true, linkcolor=black, urlcolor=blue]{hyperref}
\usepackage{setspace}
\setlength{\parskip}{5pt}
\setlength{\parindent}{0pt}


\title{Facharbeit Deutsch}
\author{Neville Brem}


\begin{document}

\maketitle

\newpage

\tableofcontents

\newpage

\section{Einleitung}
Auch politische Akteure haben erkannt, wie einfach und effektiv man die sozialen Medien für mehr Reichweite nutzen kann. So hat auch die SVP einen TikTok-Account (\href{https://www.tiktok.com/@svpch}{@svpch}). Diese Facharbeit sollte analysieren, womit die SVP ihre politische Agenda im Hinsicht auf die (zum Zeitpunkt der Verfassung dieser Arbeit) bevorstehende Autobahnausbau-Initiative verbreitet. Marcel Dettling, seit 2024 der Präsident der SVP Schweiz und Moderator des Podcast "Dütsch Dütlich Dettling", hat eine grosse Präsenz auf der Plattform lädt regelmässig Videos zur Kampagne hoch.

Diese Arbeit sollte beleuchten, mit welchen Mitteln der Einschüchterung Marcel Dettling es erreichen will, die Bevölkerung auf seine Seite zu ziehen. Dazu habe ich einen spezifischen TikTok-Clip\footnote{@svpch auf \href{https://www.tiktok.com/@svpch/video/7436069684971195680}{TikTok}} ausgewählt, in dem Dettling ausgiebig über das Thema spricht. Das Video ist ebenfalls im Ornder als .mp4-Datei\footnote{\href{https://eduzh-my.sharepoint.com/:v:/r/personal/neville_brem_mng_ch/Documents/HS24/Deutsch/Rhetorik/Facharbeit/Quellen/Video/tiktok_svpch_7432338486931246369.mp4?csf=1&web=1&e=eyQNpr}{OneDrive}} auffindbar, falls das Video verschwinden sollte.

\end{document}